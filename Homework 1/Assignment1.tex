% Options for packages loaded elsewhere
\PassOptionsToPackage{unicode}{hyperref}
\PassOptionsToPackage{hyphens}{url}
%
\documentclass[
]{article}
\usepackage{amsmath,amssymb}
\usepackage{iftex}
\ifPDFTeX
  \usepackage[T1]{fontenc}
  \usepackage[utf8]{inputenc}
  \usepackage{textcomp} % provide euro and other symbols
\else % if luatex or xetex
  \usepackage{unicode-math} % this also loads fontspec
  \defaultfontfeatures{Scale=MatchLowercase}
  \defaultfontfeatures[\rmfamily]{Ligatures=TeX,Scale=1}
\fi
\usepackage{lmodern}
\ifPDFTeX\else
  % xetex/luatex font selection
\fi
% Use upquote if available, for straight quotes in verbatim environments
\IfFileExists{upquote.sty}{\usepackage{upquote}}{}
\IfFileExists{microtype.sty}{% use microtype if available
  \usepackage[]{microtype}
  \UseMicrotypeSet[protrusion]{basicmath} % disable protrusion for tt fonts
}{}
\makeatletter
\@ifundefined{KOMAClassName}{% if non-KOMA class
  \IfFileExists{parskip.sty}{%
    \usepackage{parskip}
  }{% else
    \setlength{\parindent}{0pt}
    \setlength{\parskip}{6pt plus 2pt minus 1pt}}
}{% if KOMA class
  \KOMAoptions{parskip=half}}
\makeatother
\usepackage{xcolor}
\usepackage[margin=1in]{geometry}
\usepackage{color}
\usepackage{fancyvrb}
\newcommand{\VerbBar}{|}
\newcommand{\VERB}{\Verb[commandchars=\\\{\}]}
\DefineVerbatimEnvironment{Highlighting}{Verbatim}{commandchars=\\\{\}}
% Add ',fontsize=\small' for more characters per line
\usepackage{framed}
\definecolor{shadecolor}{RGB}{248,248,248}
\newenvironment{Shaded}{\begin{snugshade}}{\end{snugshade}}
\newcommand{\AlertTok}[1]{\textcolor[rgb]{0.94,0.16,0.16}{#1}}
\newcommand{\AnnotationTok}[1]{\textcolor[rgb]{0.56,0.35,0.01}{\textbf{\textit{#1}}}}
\newcommand{\AttributeTok}[1]{\textcolor[rgb]{0.13,0.29,0.53}{#1}}
\newcommand{\BaseNTok}[1]{\textcolor[rgb]{0.00,0.00,0.81}{#1}}
\newcommand{\BuiltInTok}[1]{#1}
\newcommand{\CharTok}[1]{\textcolor[rgb]{0.31,0.60,0.02}{#1}}
\newcommand{\CommentTok}[1]{\textcolor[rgb]{0.56,0.35,0.01}{\textit{#1}}}
\newcommand{\CommentVarTok}[1]{\textcolor[rgb]{0.56,0.35,0.01}{\textbf{\textit{#1}}}}
\newcommand{\ConstantTok}[1]{\textcolor[rgb]{0.56,0.35,0.01}{#1}}
\newcommand{\ControlFlowTok}[1]{\textcolor[rgb]{0.13,0.29,0.53}{\textbf{#1}}}
\newcommand{\DataTypeTok}[1]{\textcolor[rgb]{0.13,0.29,0.53}{#1}}
\newcommand{\DecValTok}[1]{\textcolor[rgb]{0.00,0.00,0.81}{#1}}
\newcommand{\DocumentationTok}[1]{\textcolor[rgb]{0.56,0.35,0.01}{\textbf{\textit{#1}}}}
\newcommand{\ErrorTok}[1]{\textcolor[rgb]{0.64,0.00,0.00}{\textbf{#1}}}
\newcommand{\ExtensionTok}[1]{#1}
\newcommand{\FloatTok}[1]{\textcolor[rgb]{0.00,0.00,0.81}{#1}}
\newcommand{\FunctionTok}[1]{\textcolor[rgb]{0.13,0.29,0.53}{\textbf{#1}}}
\newcommand{\ImportTok}[1]{#1}
\newcommand{\InformationTok}[1]{\textcolor[rgb]{0.56,0.35,0.01}{\textbf{\textit{#1}}}}
\newcommand{\KeywordTok}[1]{\textcolor[rgb]{0.13,0.29,0.53}{\textbf{#1}}}
\newcommand{\NormalTok}[1]{#1}
\newcommand{\OperatorTok}[1]{\textcolor[rgb]{0.81,0.36,0.00}{\textbf{#1}}}
\newcommand{\OtherTok}[1]{\textcolor[rgb]{0.56,0.35,0.01}{#1}}
\newcommand{\PreprocessorTok}[1]{\textcolor[rgb]{0.56,0.35,0.01}{\textit{#1}}}
\newcommand{\RegionMarkerTok}[1]{#1}
\newcommand{\SpecialCharTok}[1]{\textcolor[rgb]{0.81,0.36,0.00}{\textbf{#1}}}
\newcommand{\SpecialStringTok}[1]{\textcolor[rgb]{0.31,0.60,0.02}{#1}}
\newcommand{\StringTok}[1]{\textcolor[rgb]{0.31,0.60,0.02}{#1}}
\newcommand{\VariableTok}[1]{\textcolor[rgb]{0.00,0.00,0.00}{#1}}
\newcommand{\VerbatimStringTok}[1]{\textcolor[rgb]{0.31,0.60,0.02}{#1}}
\newcommand{\WarningTok}[1]{\textcolor[rgb]{0.56,0.35,0.01}{\textbf{\textit{#1}}}}
\usepackage{graphicx}
\makeatletter
\def\maxwidth{\ifdim\Gin@nat@width>\linewidth\linewidth\else\Gin@nat@width\fi}
\def\maxheight{\ifdim\Gin@nat@height>\textheight\textheight\else\Gin@nat@height\fi}
\makeatother
% Scale images if necessary, so that they will not overflow the page
% margins by default, and it is still possible to overwrite the defaults
% using explicit options in \includegraphics[width, height, ...]{}
\setkeys{Gin}{width=\maxwidth,height=\maxheight,keepaspectratio}
% Set default figure placement to htbp
\makeatletter
\def\fps@figure{htbp}
\makeatother
\setlength{\emergencystretch}{3em} % prevent overfull lines
\providecommand{\tightlist}{%
  \setlength{\itemsep}{0pt}\setlength{\parskip}{0pt}}
\setcounter{secnumdepth}{-\maxdimen} % remove section numbering
\ifLuaTeX
  \usepackage{selnolig}  % disable illegal ligatures
\fi
\usepackage{bookmark}
\IfFileExists{xurl.sty}{\usepackage{xurl}}{} % add URL line breaks if available
\urlstyle{same}
\hypersetup{
  hidelinks,
  pdfcreator={LaTeX via pandoc}}

\author{}
\date{\vspace{-2.5em}}

\begin{document}

Q2.1 (a)

\begin{Shaded}
\begin{Highlighting}[]
\CommentTok{\# 95th percentile}
\FunctionTok{qnorm}\NormalTok{(}\FloatTok{0.95}\NormalTok{, }\AttributeTok{mean=}\DecValTok{10}\NormalTok{, }\AttributeTok{sd=}\DecValTok{3}\NormalTok{)}
\end{Highlighting}
\end{Shaded}

\begin{verbatim}
## [1] 14.93456
\end{verbatim}

\begin{Shaded}
\begin{Highlighting}[]
\CommentTok{\# 99th percentile}
\FunctionTok{qnorm}\NormalTok{(}\FloatTok{0.99}\NormalTok{, }\AttributeTok{mean=}\DecValTok{10}\NormalTok{, }\AttributeTok{sd=}\DecValTok{3}\NormalTok{)}
\end{Highlighting}
\end{Shaded}

\begin{verbatim}
## [1] 16.97904
\end{verbatim}

Q2.1 (b)

\begin{Shaded}
\begin{Highlighting}[]
\CommentTok{\# 95th percentile of the t{-}distribution with 10 degrees of freedom}
\FunctionTok{qt}\NormalTok{(}\FloatTok{0.95}\NormalTok{, }\AttributeTok{df=}\DecValTok{10}\NormalTok{)}
\end{Highlighting}
\end{Shaded}

\begin{verbatim}
## [1] 1.812461
\end{verbatim}

\begin{Shaded}
\begin{Highlighting}[]
\CommentTok{\# 99th percentile of the t{-}distribution with 10 degrees of freedom}
\FunctionTok{qt}\NormalTok{(}\FloatTok{0.99}\NormalTok{, }\AttributeTok{df=}\DecValTok{10}\NormalTok{)}
\end{Highlighting}
\end{Shaded}

\begin{verbatim}
## [1] 2.763769
\end{verbatim}

\begin{Shaded}
\begin{Highlighting}[]
\CommentTok{\# 95th percentile of the t{-}distribution with 25 degrees of freedom}
\FunctionTok{qt}\NormalTok{(}\FloatTok{0.95}\NormalTok{, }\AttributeTok{df=}\DecValTok{25}\NormalTok{)}
\end{Highlighting}
\end{Shaded}

\begin{verbatim}
## [1] 1.708141
\end{verbatim}

\begin{Shaded}
\begin{Highlighting}[]
\CommentTok{\# 99th percentile of the t{-}distribution with 25 degrees of freedom}
\FunctionTok{qt}\NormalTok{(}\FloatTok{0.99}\NormalTok{, }\AttributeTok{df=}\DecValTok{25}\NormalTok{)}
\end{Highlighting}
\end{Shaded}

\begin{verbatim}
## [1] 2.485107
\end{verbatim}

Q2.1 (c)

\begin{Shaded}
\begin{Highlighting}[]
\CommentTok{\# 95th percentile of the chi{-}squared distribution with 1 degrees of freedom}
\FunctionTok{qchisq}\NormalTok{(}\FloatTok{0.95}\NormalTok{, }\AttributeTok{df=}\DecValTok{1}\NormalTok{)}
\end{Highlighting}
\end{Shaded}

\begin{verbatim}
## [1] 3.841459
\end{verbatim}

\begin{Shaded}
\begin{Highlighting}[]
\CommentTok{\# 99th percentile of the chi{-}squared distribution with 1 degrees of freedom}
\FunctionTok{qchisq}\NormalTok{(}\FloatTok{0.99}\NormalTok{, }\AttributeTok{df=}\DecValTok{1}\NormalTok{)}
\end{Highlighting}
\end{Shaded}

\begin{verbatim}
## [1] 6.634897
\end{verbatim}

\begin{Shaded}
\begin{Highlighting}[]
\CommentTok{\# 95th percentile of the chi{-}squared distribution with 4 degrees of freedom}
\FunctionTok{qchisq}\NormalTok{(}\FloatTok{0.95}\NormalTok{, }\AttributeTok{df=}\DecValTok{4}\NormalTok{)}
\end{Highlighting}
\end{Shaded}

\begin{verbatim}
## [1] 9.487729
\end{verbatim}

\begin{Shaded}
\begin{Highlighting}[]
\CommentTok{\# 99th percentile of the chi{-}squared distribution with 4 degrees of freedom}
\FunctionTok{qchisq}\NormalTok{(}\FloatTok{0.99}\NormalTok{, }\AttributeTok{df=}\DecValTok{4}\NormalTok{)}
\end{Highlighting}
\end{Shaded}

\begin{verbatim}
## [1] 13.2767
\end{verbatim}

\begin{Shaded}
\begin{Highlighting}[]
\CommentTok{\# 95th percentile of the chi{-}squared distribution with 10 degrees of freedom}
\FunctionTok{qchisq}\NormalTok{(}\FloatTok{0.95}\NormalTok{, }\AttributeTok{df=}\DecValTok{10}\NormalTok{)}
\end{Highlighting}
\end{Shaded}

\begin{verbatim}
## [1] 18.30704
\end{verbatim}

\begin{Shaded}
\begin{Highlighting}[]
\CommentTok{\# 99th percentile of the chi{-}squared distribution with 10 degrees of freedom}
\FunctionTok{qchisq}\NormalTok{(}\FloatTok{0.99}\NormalTok{, }\AttributeTok{df=}\DecValTok{10}\NormalTok{)}
\end{Highlighting}
\end{Shaded}

\begin{verbatim}
## [1] 23.20925
\end{verbatim}

Q2.1 (d)

\begin{Shaded}
\begin{Highlighting}[]
\CommentTok{\# 95th percentile of the F{-}distribution with 2 and 10 degrees of freedom}
\FunctionTok{qf}\NormalTok{(}\FloatTok{0.95}\NormalTok{, }\AttributeTok{df1=}\DecValTok{2}\NormalTok{, }\AttributeTok{df2=}\DecValTok{10}\NormalTok{)}
\end{Highlighting}
\end{Shaded}

\begin{verbatim}
## [1] 4.102821
\end{verbatim}

\begin{Shaded}
\begin{Highlighting}[]
\CommentTok{\# 99th percentile of the F{-}distribution with 2 and 10 degrees of freedom}
\FunctionTok{qf}\NormalTok{(}\FloatTok{0.99}\NormalTok{, }\AttributeTok{df1=}\DecValTok{2}\NormalTok{, }\AttributeTok{df2=}\DecValTok{10}\NormalTok{)}
\end{Highlighting}
\end{Shaded}

\begin{verbatim}
## [1] 7.559432
\end{verbatim}

\begin{Shaded}
\begin{Highlighting}[]
\CommentTok{\# 95th percentile of the F{-}distribution with 4 and 10 degrees of freedom}
\FunctionTok{qf}\NormalTok{(}\FloatTok{0.95}\NormalTok{, }\AttributeTok{df1=}\DecValTok{4}\NormalTok{, }\AttributeTok{df2=}\DecValTok{10}\NormalTok{)}
\end{Highlighting}
\end{Shaded}

\begin{verbatim}
## [1] 3.47805
\end{verbatim}

\begin{Shaded}
\begin{Highlighting}[]
\CommentTok{\# 99th percentile of the F{-}distribution with 4 and 10 degrees of freedom}
\FunctionTok{qf}\NormalTok{(}\FloatTok{0.99}\NormalTok{, }\AttributeTok{df1=}\DecValTok{4}\NormalTok{, }\AttributeTok{df2=}\DecValTok{10}\NormalTok{)}
\end{Highlighting}
\end{Shaded}

\begin{verbatim}
## [1] 5.994339
\end{verbatim}

Q2.2 (a)

\begin{Shaded}
\begin{Highlighting}[]
\CommentTok{\# 95th percentile of square of standard normal distribution}
\FunctionTok{qnorm}\NormalTok{(}\FloatTok{0.95}\NormalTok{, }\AttributeTok{mean =} \DecValTok{0}\NormalTok{, }\AttributeTok{sd=}\DecValTok{1}\NormalTok{)}\SpecialCharTok{\^{}}\DecValTok{2}
\end{Highlighting}
\end{Shaded}

\begin{verbatim}
## [1] 2.705543
\end{verbatim}

\begin{Shaded}
\begin{Highlighting}[]
\CommentTok{\# 90th percentile of chi{-}squared distribution with 1 degree of freedom}
\FunctionTok{qchisq}\NormalTok{(}\FloatTok{0.90}\NormalTok{, }\AttributeTok{df=}\DecValTok{1}\NormalTok{)}
\end{Highlighting}
\end{Shaded}

\begin{verbatim}
## [1] 2.705543
\end{verbatim}

\begin{Shaded}
\begin{Highlighting}[]
\CommentTok{\# 97.5th percentile of square of standard normal distribution}
\FunctionTok{qnorm}\NormalTok{(}\FloatTok{0.975}\NormalTok{, }\AttributeTok{mean =} \DecValTok{0}\NormalTok{, }\AttributeTok{sd=}\DecValTok{1}\NormalTok{)}\SpecialCharTok{\^{}}\DecValTok{2}
\end{Highlighting}
\end{Shaded}

\begin{verbatim}
## [1] 3.841459
\end{verbatim}

\begin{Shaded}
\begin{Highlighting}[]
\CommentTok{\# 95th percentile of chi{-}squared distribution with 1 degree of freedom}
\FunctionTok{qchisq}\NormalTok{(}\FloatTok{0.95}\NormalTok{, }\AttributeTok{df=}\DecValTok{1}\NormalTok{)}
\end{Highlighting}
\end{Shaded}

\begin{verbatim}
## [1] 3.841459
\end{verbatim}

\begin{Shaded}
\begin{Highlighting}[]
\CommentTok{\# 98.75th percentile of square of standard normal distribution}
\FunctionTok{qnorm}\NormalTok{(}\FloatTok{0.9875}\NormalTok{, }\AttributeTok{mean =} \DecValTok{0}\NormalTok{, }\AttributeTok{sd=}\DecValTok{1}\NormalTok{)}\SpecialCharTok{\^{}}\DecValTok{2}
\end{Highlighting}
\end{Shaded}

\begin{verbatim}
## [1] 5.023886
\end{verbatim}

\begin{Shaded}
\begin{Highlighting}[]
\CommentTok{\# 97.5th percentile of chi{-}squared distribution with 1 degree of freedom}
\FunctionTok{qchisq}\NormalTok{(}\FloatTok{0.975}\NormalTok{, }\AttributeTok{df=}\DecValTok{1}\NormalTok{)}
\end{Highlighting}
\end{Shaded}

\begin{verbatim}
## [1] 5.023886
\end{verbatim}

We can see from the following results that:

\begin{enumerate}
\def\labelenumi{\arabic{enumi}.}
\item
  the 90th percentile of the chi-squared distribution is equal to the
  95th percentile of the square of the standard normal distribution.
\item
  the 95th percentile of the chi-squared distribution is equal to the
  97.5th percentile of the square of the standard normal distribution.
\item
  the 97.5th percentile of the chi-squared distribution is equal to the
  98.75th percentile of the square of the standard normal distribution.
\end{enumerate}

Q2.2 (b)

\begin{Shaded}
\begin{Highlighting}[]
\NormalTok{v1 }\OtherTok{\textless{}{-}} \DecValTok{1}
\NormalTok{v2 }\OtherTok{\textless{}{-}} \DecValTok{20}

\CommentTok{\# 95th percentile of square of t distribution}
\FunctionTok{qt}\NormalTok{(}\FloatTok{0.95}\NormalTok{, }\AttributeTok{df=}\NormalTok{v2)}\SpecialCharTok{\^{}}\DecValTok{2}
\end{Highlighting}
\end{Shaded}

\begin{verbatim}
## [1] 2.974653
\end{verbatim}

\begin{Shaded}
\begin{Highlighting}[]
\CommentTok{\# 90th percentile of f distribution (1, v2)}
\FunctionTok{qf}\NormalTok{(}\FloatTok{0.90}\NormalTok{, }\AttributeTok{df1=}\NormalTok{v1, }\AttributeTok{df2=}\NormalTok{v2)}
\end{Highlighting}
\end{Shaded}

\begin{verbatim}
## [1] 2.974653
\end{verbatim}

\begin{Shaded}
\begin{Highlighting}[]
\CommentTok{\# 97.5th percentile of square of t distribution}
\FunctionTok{qt}\NormalTok{(}\FloatTok{0.975}\NormalTok{, }\AttributeTok{df=}\NormalTok{v2)}\SpecialCharTok{\^{}}\DecValTok{2}
\end{Highlighting}
\end{Shaded}

\begin{verbatim}
## [1] 4.351244
\end{verbatim}

\begin{Shaded}
\begin{Highlighting}[]
\CommentTok{\# 95th percentile of f distribution (1, v2)}
\FunctionTok{qf}\NormalTok{(}\FloatTok{0.95}\NormalTok{, }\AttributeTok{df1=}\NormalTok{v1, }\AttributeTok{df2=}\NormalTok{v2)}
\end{Highlighting}
\end{Shaded}

\begin{verbatim}
## [1] 4.351244
\end{verbatim}

\begin{Shaded}
\begin{Highlighting}[]
\CommentTok{\# 98.75th percentile of square of t distribution}
\FunctionTok{qt}\NormalTok{(}\FloatTok{0.9875}\NormalTok{, }\AttributeTok{df=}\NormalTok{v2)}\SpecialCharTok{\^{}}\DecValTok{2}
\end{Highlighting}
\end{Shaded}

\begin{verbatim}
## [1] 5.871494
\end{verbatim}

\begin{Shaded}
\begin{Highlighting}[]
\CommentTok{\# 97.5th percentile of f distribution (1, v2)}
\FunctionTok{qf}\NormalTok{(}\FloatTok{0.975}\NormalTok{, }\AttributeTok{df1=}\NormalTok{v1, }\AttributeTok{df2=}\NormalTok{v2)}
\end{Highlighting}
\end{Shaded}

\begin{verbatim}
## [1] 5.871494
\end{verbatim}

We can see from the following results that:

\begin{enumerate}
\def\labelenumi{\arabic{enumi}.}
\item
  the 90th percentile of the F distribution (1, v2) is equal to the 95th
  percentile of the square of the t distribution (v2).
\item
  the 95th percentile of the F distribution (1, v2) is equal to the
  97.5th percentile of the square of the t distribution (v2).
\item
  the 97.5th percentile of the F distribution (1, v2) is equal to the
  98.75th percentile of the square of the t distribution (v2).
\end{enumerate}

Q2.4 (a)

\begin{Shaded}
\begin{Highlighting}[]
\NormalTok{weekof }\OtherTok{\textless{}{-}} \FunctionTok{c}\NormalTok{(}\StringTok{"January 30"}\NormalTok{, }\StringTok{"June 39"}\NormalTok{, }\StringTok{"March 2"}\NormalTok{, }\StringTok{"October 26"}\NormalTok{, }\StringTok{"February 7"}\NormalTok{)}
\NormalTok{no\_of\_cars\_sold\_y }\OtherTok{\textless{}{-}} \FunctionTok{c}\NormalTok{(}\DecValTok{20}\NormalTok{, }\DecValTok{18}\NormalTok{, }\DecValTok{10}\NormalTok{, }\DecValTok{6}\NormalTok{, }\DecValTok{11}\NormalTok{)}
\NormalTok{ave\_no\_of\_salesppl\_x }\OtherTok{\textless{}{-}} \FunctionTok{c}\NormalTok{(}\DecValTok{6}\NormalTok{, }\DecValTok{6}\NormalTok{, }\DecValTok{4}\NormalTok{, }\DecValTok{2}\NormalTok{, }\DecValTok{3}\NormalTok{)}

\CommentTok{\# create matrix with the data}
\NormalTok{data }\OtherTok{\textless{}{-}} \FunctionTok{cbind}\NormalTok{(weekof, no\_of\_cars\_sold\_y, ave\_no\_of\_salesppl\_x)}

\CommentTok{\# create data frame}
\NormalTok{df }\OtherTok{\textless{}{-}} \FunctionTok{as.data.frame}\NormalTok{(data)}
\FunctionTok{transform}\NormalTok{(df, }\AttributeTok{no\_of\_cars\_sold\_y =} \FunctionTok{as.numeric}\NormalTok{(no\_of\_cars\_sold\_y), }\AttributeTok{ave\_no\_of\_salesppl\_x =} \FunctionTok{as.numeric}\NormalTok{(ave\_no\_of\_salesppl\_x))}
\end{Highlighting}
\end{Shaded}

\begin{verbatim}
##       weekof no_of_cars_sold_y ave_no_of_salesppl_x
## 1 January 30                20                    6
## 2    June 39                18                    6
## 3    March 2                10                    4
## 4 October 26                 6                    2
## 5 February 7                11                    3
\end{verbatim}

\begin{Shaded}
\begin{Highlighting}[]
\CommentTok{\# construct scatterplot of y against x, where y is the number of cars sold and x is the average number of salespeople}
\FunctionTok{plot}\NormalTok{(ave\_no\_of\_salesppl\_x, no\_of\_cars\_sold\_y, }\AttributeTok{xlab=}\StringTok{"Ave \# of salespeople"}\NormalTok{, }\AttributeTok{ylab=}\StringTok{"\# of cars sold"}\NormalTok{, }\AttributeTok{main=}\StringTok{"Scatterplot of \# of cars sold against ave \# of salespeople"}\NormalTok{)}
\end{Highlighting}
\end{Shaded}

\includegraphics{Assignment1_files/figure-latex/unnamed-chunk-9-1.pdf}
Q2.4 (b)

\begin{Shaded}
\begin{Highlighting}[]
\CommentTok{\# estimate intercept using Method of least squares}
\NormalTok{xbar }\OtherTok{=} \FunctionTok{mean}\NormalTok{(ave\_no\_of\_salesppl\_x)}
\NormalTok{ybar }\OtherTok{=} \FunctionTok{mean}\NormalTok{(no\_of\_cars\_sold\_y)}
\NormalTok{b1 }\OtherTok{=} \FunctionTok{sum}\NormalTok{((ave\_no\_of\_salesppl\_x }\SpecialCharTok{{-}}\NormalTok{ xbar) }\SpecialCharTok{*}\NormalTok{ (no\_of\_cars\_sold\_y }\SpecialCharTok{{-}}\NormalTok{ ybar)) }\SpecialCharTok{/} \FunctionTok{sum}\NormalTok{((ave\_no\_of\_salesppl\_x }\SpecialCharTok{{-}}\NormalTok{ xbar)}\SpecialCharTok{\^{}}\DecValTok{2}\NormalTok{)}
\NormalTok{b0 }\OtherTok{=}\NormalTok{ ybar }\SpecialCharTok{{-}}\NormalTok{ b1 }\SpecialCharTok{*}\NormalTok{ xbar}
\FunctionTok{cat}\NormalTok{(}\StringTok{"The estimated intercept:"}\NormalTok{, b0, }\StringTok{"}\SpecialCharTok{\textbackslash{}n}\StringTok{"}\NormalTok{)}
\end{Highlighting}
\end{Shaded}

\begin{verbatim}
## The estimated intercept: -0.125
\end{verbatim}

\begin{Shaded}
\begin{Highlighting}[]
\FunctionTok{cat}\NormalTok{(}\StringTok{"The estimated slope of the least squares line:"}\NormalTok{, b1, }\StringTok{"}\SpecialCharTok{\textbackslash{}n}\StringTok{"}\NormalTok{)}
\end{Highlighting}
\end{Shaded}

\begin{verbatim}
## The estimated slope of the least squares line: 3.125
\end{verbatim}

\begin{Shaded}
\begin{Highlighting}[]
\CommentTok{\# check using built in function \textquotesingle{}lm\textquotesingle{}}
\NormalTok{model }\OtherTok{\textless{}{-}} \FunctionTok{lm}\NormalTok{(no\_of\_cars\_sold\_y }\SpecialCharTok{\textasciitilde{}}\NormalTok{ ave\_no\_of\_salesppl\_x)}

\CommentTok{\# print the summary of the model}
\FunctionTok{summary}\NormalTok{(model)}
\end{Highlighting}
\end{Shaded}

\begin{verbatim}
## 
## Call:
## lm(formula = no_of_cars_sold_y ~ ave_no_of_salesppl_x)
## 
## Residuals:
##      1      2      3      4      5 
##  1.375 -0.625 -2.375 -0.125  1.750 
## 
## Coefficients:
##                      Estimate Std. Error t value Pr(>|t|)  
## (Intercept)           -0.1250     2.4055  -0.052    0.962  
## ave_no_of_salesppl_x   3.1250     0.5352   5.839    0.010 *
## ---
## Signif. codes:  0 '***' 0.001 '**' 0.01 '*' 0.05 '.' 0.1 ' ' 1
## 
## Residual standard error: 1.915 on 3 degrees of freedom
## Multiple R-squared:  0.9191, Adjusted R-squared:  0.8922 
## F-statistic: 34.09 on 1 and 3 DF,  p-value: 0.01001
\end{verbatim}

Q2.4 (c)

\begin{Shaded}
\begin{Highlighting}[]
\CommentTok{\# plot the least squares line}
\FunctionTok{plot}\NormalTok{(ave\_no\_of\_salesppl\_x, no\_of\_cars\_sold\_y, }\AttributeTok{xlab=}\StringTok{"Ave \# of salespeople"}\NormalTok{, }\AttributeTok{ylab=}\StringTok{"\# of cars sold"}\NormalTok{, }\AttributeTok{main=}\StringTok{"Scatterplot of \# of cars sold against ave \# of salespeople"}\NormalTok{)}
\FunctionTok{abline}\NormalTok{(model, }\AttributeTok{col=}\StringTok{"red"}\NormalTok{)}
\end{Highlighting}
\end{Shaded}

\includegraphics{Assignment1_files/figure-latex/unnamed-chunk-11-1.pdf}
Q2.4 (d)

\begin{Shaded}
\begin{Highlighting}[]
\CommentTok{\# estimate the number of cars sold when the average number of salespeople is 5}
\NormalTok{x }\OtherTok{=} \DecValTok{5}
\NormalTok{y }\OtherTok{=}\NormalTok{ b0 }\SpecialCharTok{+}\NormalTok{ b1 }\SpecialCharTok{*}\NormalTok{ x}
\FunctionTok{cat}\NormalTok{(}\StringTok{"The estimated number of cars sold when the average number of salespeople is 5:"}\NormalTok{, y, }\StringTok{"}\SpecialCharTok{\textbackslash{}n}\StringTok{"}\NormalTok{)}
\end{Highlighting}
\end{Shaded}

\begin{verbatim}
## The estimated number of cars sold when the average number of salespeople is 5: 15.5
\end{verbatim}

Q2.4 (e)

\begin{Shaded}
\begin{Highlighting}[]
\CommentTok{\# Calculate the fitted values mu\_hat for each observed value of x}
\NormalTok{mu\_hat }\OtherTok{=}\NormalTok{ b0 }\SpecialCharTok{+}\NormalTok{ b1 }\SpecialCharTok{*}\NormalTok{ ave\_no\_of\_salesppl\_x}

\CommentTok{\# Calculate the residuals e\_i for each observed value of x}
\NormalTok{e\_i }\OtherTok{=}\NormalTok{ no\_of\_cars\_sold\_y }\SpecialCharTok{{-}}\NormalTok{ mu\_hat}

\CommentTok{\# Plot the residuals against the fitted values}
\FunctionTok{plot}\NormalTok{(mu\_hat, e\_i, }\AttributeTok{xlab=}\StringTok{"Fitted values"}\NormalTok{, }\AttributeTok{ylab=}\StringTok{"Residuals"}\NormalTok{, }\AttributeTok{main=}\StringTok{"Residuals against Fitted values"}\NormalTok{)}
\end{Highlighting}
\end{Shaded}

\includegraphics{Assignment1_files/figure-latex/unnamed-chunk-13-1.pdf}
Q2.4 (f)

\begin{Shaded}
\begin{Highlighting}[]
\CommentTok{\# Calculate manually an estimate of the variance sigma square}
\NormalTok{n }\OtherTok{=} \FunctionTok{length}\NormalTok{(no\_of\_cars\_sold\_y)}
\NormalTok{rss }\OtherTok{=} \FunctionTok{sum}\NormalTok{(e\_i}\SpecialCharTok{\^{}}\DecValTok{2}\NormalTok{)}
\NormalTok{estimate\_variance }\OtherTok{=}\NormalTok{ rss }\SpecialCharTok{/}\NormalTok{ (n }\SpecialCharTok{{-}} \DecValTok{2}\NormalTok{)}
\FunctionTok{cat}\NormalTok{(}\StringTok{"The estimated variance:"}\NormalTok{, estimate\_variance, }\StringTok{"}\SpecialCharTok{\textbackslash{}n}\StringTok{"}\NormalTok{)}
\end{Highlighting}
\end{Shaded}

\begin{verbatim}
## The estimated variance: 3.666667
\end{verbatim}

Q2.6 (a)

\begin{Shaded}
\begin{Highlighting}[]
\CommentTok{\# import data}
\NormalTok{q6data }\OtherTok{=} \FunctionTok{read.table}\NormalTok{(}\AttributeTok{file=}\StringTok{\textquotesingle{}hooker.txt\textquotesingle{}}\NormalTok{, }\AttributeTok{header =}\NormalTok{ T)}
\NormalTok{q6data}
\end{Highlighting}
\end{Shaded}

\begin{verbatim}
##       BT     AP
## 1  210.8 29.211
## 2  210.2 28.559
## 3  208.4 27.972
## 4  202.5 24.697
## 5  200.6 23.726
## 6  200.1 23.369
## 7  199.5 23.030
## 8  197.0 21.892
## 9  196.4 21.928
## 10 196.3 21.654
## 11 195.6 21.605
## 12 193.4 20.480
## 13 193.6 20.212
## 14 191.4 19.758
## 15 191.1 19.490
## 16 190.6 19.386
## 17 189.5 18.869
## 18 188.8 18.356
## 19 188.5 18.507
## 20 185.7 17.267
## 21 186.0 17.221
## 22 185.6 17.062
## 23 184.1 16.959
## 24 184.6 16.881
## 25 184.1 16.817
## 26 183.2 16.385
## 27 182.4 16.235
## 28 181.9 16.106
## 29 181.9 15.928
## 30 181.0 15.919
## 31 180.6 15.376
\end{verbatim}

\begin{Shaded}
\begin{Highlighting}[]
\CommentTok{\# Plot TEMP against AP, where BT is the temperature and AP is the air pressure}
\FunctionTok{plot}\NormalTok{(q6data}\SpecialCharTok{$}\NormalTok{AP, q6data}\SpecialCharTok{$}\NormalTok{BT, }\AttributeTok{xlab=}\StringTok{"AP"}\NormalTok{, }\AttributeTok{ylab=}\StringTok{"TEMP"}\NormalTok{, }\AttributeTok{main=}\StringTok{"Scatterplot of TEMP against AP"}\NormalTok{)}
\end{Highlighting}
\end{Shaded}

\includegraphics{Assignment1_files/figure-latex/unnamed-chunk-17-1.pdf}
- A Linear model seems appropriate as the relationship between TEMP and
AP seems to be linear.

Q2.6 (b)

\begin{Shaded}
\begin{Highlighting}[]
\NormalTok{x }\OtherTok{=} \DecValTok{100} \SpecialCharTok{*} \FunctionTok{log}\NormalTok{(q6data}\SpecialCharTok{$}\NormalTok{AP, }\AttributeTok{base =} \FunctionTok{exp}\NormalTok{(}\DecValTok{1}\NormalTok{))}

\CommentTok{\# plot the scatterplot of TEMP against x}
\FunctionTok{plot}\NormalTok{(x, q6data}\SpecialCharTok{$}\NormalTok{BT, }\AttributeTok{xlab=}\StringTok{"100 * ln(AP)"}\NormalTok{, }\AttributeTok{ylab=}\StringTok{"TEMP"}\NormalTok{, }\AttributeTok{main=}\StringTok{"Scatterplot of TEMP against 100 * ln(AP)"}\NormalTok{)}
\end{Highlighting}
\end{Shaded}

\includegraphics{Assignment1_files/figure-latex/unnamed-chunk-18-1.pdf}
Q2.6 (c)

\begin{Shaded}
\begin{Highlighting}[]
\NormalTok{xbar }\OtherTok{=} \FunctionTok{mean}\NormalTok{(q6data}\SpecialCharTok{$}\NormalTok{AP)}
\NormalTok{ybar }\OtherTok{=} \FunctionTok{mean}\NormalTok{(q6data}\SpecialCharTok{$}\NormalTok{BT)}
\NormalTok{b1 }\OtherTok{=} \FunctionTok{sum}\NormalTok{((q6data}\SpecialCharTok{$}\NormalTok{AP }\SpecialCharTok{{-}}\NormalTok{ xbar) }\SpecialCharTok{*}\NormalTok{ (q6data}\SpecialCharTok{$}\NormalTok{BT }\SpecialCharTok{{-}}\NormalTok{ ybar)) }\SpecialCharTok{/} \FunctionTok{sum}\NormalTok{((q6data}\SpecialCharTok{$}\NormalTok{AP }\SpecialCharTok{{-}}\NormalTok{ xbar)}\SpecialCharTok{\^{}}\DecValTok{2}\NormalTok{)}
\NormalTok{b0 }\OtherTok{=}\NormalTok{ ybar }\SpecialCharTok{{-}}\NormalTok{ b1 }\SpecialCharTok{*}\NormalTok{ xbar}
\FunctionTok{cat}\NormalTok{(}\StringTok{"The estimated intercept:"}\NormalTok{, b0, }\StringTok{"}\SpecialCharTok{\textbackslash{}n}\StringTok{"}\NormalTok{)}
\end{Highlighting}
\end{Shaded}

\begin{verbatim}
## The estimated intercept: 146.6729
\end{verbatim}

\begin{Shaded}
\begin{Highlighting}[]
\FunctionTok{cat}\NormalTok{(}\StringTok{"The estimated slope:"}\NormalTok{, b1, }\StringTok{"}\SpecialCharTok{\textbackslash{}n}\StringTok{"}\NormalTok{)}
\end{Highlighting}
\end{Shaded}

\begin{verbatim}
## The estimated slope: 2.252596
\end{verbatim}

\begin{Shaded}
\begin{Highlighting}[]
\CommentTok{\# check using built in function \textquotesingle{}lm\textquotesingle{}}
\NormalTok{q6model }\OtherTok{\textless{}{-}} \FunctionTok{lm}\NormalTok{(q6data}\SpecialCharTok{$}\NormalTok{BT }\SpecialCharTok{\textasciitilde{}}\NormalTok{ q6data}\SpecialCharTok{$}\NormalTok{AP)}

\CommentTok{\# print the summary of the model}
\FunctionTok{summary}\NormalTok{(q6model)}
\end{Highlighting}
\end{Shaded}

\begin{verbatim}
## 
## Call:
## lm(formula = q6data$BT ~ q6data$AP)
## 
## Residuals:
##     Min      1Q  Median      3Q     Max 
## -1.6735 -0.6805  0.2203  0.5296  1.3976 
## 
## Coefficients:
##              Estimate Std. Error t value Pr(>|t|)    
## (Intercept) 146.67290    0.77641  188.91   <2e-16 ***
## q6data$AP     2.25260    0.03809   59.14   <2e-16 ***
## ---
## Signif. codes:  0 '***' 0.001 '**' 0.01 '*' 0.05 '.' 0.1 ' ' 1
## 
## Residual standard error: 0.806 on 29 degrees of freedom
## Multiple R-squared:  0.9918, Adjusted R-squared:  0.9915 
## F-statistic:  3498 on 1 and 29 DF,  p-value: < 2.2e-16
\end{verbatim}

\begin{Shaded}
\begin{Highlighting}[]
\CommentTok{\# plot the least squares line}
\FunctionTok{plot}\NormalTok{(q6data}\SpecialCharTok{$}\NormalTok{AP, q6data}\SpecialCharTok{$}\NormalTok{BT, }\AttributeTok{xlab=}\StringTok{"AP"}\NormalTok{, }\AttributeTok{ylab=}\StringTok{"TEMP"}\NormalTok{, }\AttributeTok{main=}\StringTok{"Scatterplot of TEMP against AP"}\NormalTok{)}
\FunctionTok{abline}\NormalTok{(q6model, }\AttributeTok{col=}\StringTok{"red"}\NormalTok{)}
\end{Highlighting}
\end{Shaded}

\includegraphics{Assignment1_files/figure-latex/unnamed-chunk-20-1.pdf}

\end{document}
